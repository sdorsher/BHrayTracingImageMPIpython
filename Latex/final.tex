\documentclass{article}
\usepackage{listings,graphicx,amsmath}
\begin{document}
\title{A simulated telescope's view of the sky in the presence of a Schwarzschild blackhole}
\author{Steven Dorsher}

\maketitle

\section{Introduction}
With the LIGO detection, many people are interested in the question of
how blackholes make the sky appear to the external
observer. Understanding the question of how they will lens the
presence of stars behind them could help make blackholes without
accretion disks or partners indirectly observable. The problem of a
Schwarzschild blackhole in a vaccuum with light passing by it from a
static field of stars is not a particularly challenging problem and
appears to have been solved numerically in the 1970's or 1980's,
though I coudn't find a specific source. Many people since then have
solved more difficult problems for rotating, perturbed, or
asymptotically cosmological spacetimes. 

In this project, I numerically integrate the eight coupled ordinary
differential equations governing light rays (null geodesics) backward
from a telescope plane near an ordinary (no spin or charge) in a
vaccuum. I iterate over many pixels in the telescope plane, and
determine their origin on the ``sky'' at some finite distance
away. From that, I reconstruct a png file of what the sky looks like
in the presence of a Schwarzschild black hole, at the location of the
Earth, and a telescope, $700 M$ away. Henceforth, the mass of the
blackhole will simply be referred to as $M$. I will use units of $c=G=1$.


\section{The geodesic equations}

\subsection{Overview}
In general relativity, the differential equations governing the path matter or light takes in a vacuum are called the geodesic equations. They are second order rank four tensor equations, which are separable, resulting in eight coupled ordinary differential equations. The geodesic equation states that a free particle, which feels no forces other than the curvature of spacetime that causes gravity, follows a path that is the shortest path between two points. 

\subsection{Computation}

The geodesic equation is given by
\begin{equation}
\frac{d^2x^\mu}{d\lambda^2}=-\Gamma^\mu_{\alpha\beta}\frac{dx^\alpha}{d\lambda}\frac{dx^\beta}{d\lambda}
\label{eqn:geodesic}
\end{equation}
Here $x^\mu$ is the $\mu$th component of a spacetime four-vector, $\Gamma^\mu_{\alpha\beta}$ are the Christoffel connections, and $\lambda$ is an affine parameter that plays the role of proper time for a null geodesic, since a null geodesic has no rest frame. There is an implicit summation convention over repeated indices, where upper and lower indices are connected by the metric $g_{\mu\nu}$ and the inverse metric $g^{\mu\nu}$.~\cite{ref:Carrol} 

The metric determines the distance between two points in a curved
spacetime. In a flat spacetime like the one near Earth (far away from
relativistic sources), the Minkowski metric is

\begin{equation}
ds^2 = -dt^2 + dx^2 + dy^2 + dz^2 = -dt^2 + dr^2 + r^2 (d\theta^2 + \sin^2\theta d\phi^2)
\label{eqn:Minkowski}
\end{equation}

In matrix form this can be written:

\begin{equation}
\eta_{\mu\nu}=\eta^{\mu\nu}=
\begin{pmatrix}
-1 &0 &0 &0\\
0 & 1 &0 &0\\
0 & 0 &1 &0\\
0 & 0 &0 &1
\end{pmatrix}
\label{eqn:eta}
\end{equation}

In the Schwarzschild metric, 

\begin{equation}
\g_{\mu\nu}=
\begin{pmatrix}
-(1-\frac{R}{r}) &0 &0 &0\\
0 &(1-\frac{R}{r})^{-1} &0 &0\\
0 & 0 &r^2 & 0 \\
0 & 0 & 0 & r^2\sin^2\theta
\end{pmatrix}
\label{eqn:gschw}
\end{equation}
where $R=2M$ is the Schwarszchild radius of the blackhole, beyond which no light can escape.~\cite{ref:Carrol} Everything referenced in this section can be looked up in a book~\cite{ref:Carrol}, but I have rederived it in Mathematica as well. The goal was to directly obtain C++ code from Mathematica, but the scope of the project proved too large and instead I directly parallelized the python prototype. 

The inverse metric is given by
\begin{equation}
g^{\mu\nu}=
\begin{pmatrix}
-(1-\frac{R}{r})^{-1} & 0 & 0 & 0\\
0 & (1-\frac{R}{r}) & 0 & 0 \\
0 & 0 & r^{-2} &0 \\
0 & 0 & 0 & r^{-2}\csc^2\theta
\end{pmatrix}
\label{eqn:invgschw}
\end{equation}
Note that both the metric and the inverse metric are diagonal and static (independent of time) in these coordinates, symmetries that will lead to a reduction in the number of non-zero Christoffel connections. 

The Christoffel connections are computed according to the following equation:
\begin{equation}
\Gamma^\mu_{\alpha\beta}=\frac{1}{2}g^{\mu\gamma}(\partial_\beta g_{\gamma\alpha} + \partial_\alpha g_{\gamma\beta} - \partial_\gamma g_{\alpha\beta})
\label{eqn:Gamma}
\end{equation}
Where $\partial_\gamma=\frac{\partial}{\partial x^\gamma}$ for
four-coordinates $x$.~\cite{ref:Carrol}  These result in 12 independent Christoffel symbols for the Schwarzschild metric, dependant upon functions of the coordinates, except time. 

Setting the four-velocity $u^\mu = \frac{dx^\mu}{d\lambda}$. it is
possible to split the four components of the geodesic equation
(Equation~\ref{eqn:geodesic}) into eight coupled ordinary differential
equations using these Christoffel symbols. The results are given below. 


\subsection{Result}




\section{The adaptive fourth order Runga-Kutta method}
\subsection{Method}
\subsection{Testing using a simple harmonic oscillator ODE}

\section{Testing the geodesic equations with orbits with elipticity}
\subsection{Expected characteristics of these orbits}
\subsection{Initial data}
\subsection{Results}

\section{Null geodesic initial data}
\subsection{Definition of a null geodesic}
\subsection{Perpendicularity to the origin plane}
\subsection{The resulting constraints}
\subsection{End conditions: inside the horizon or outside the sky radius}
\subsection{The resulting orbits} %100 integrations plotted with splot

\section{Generating a telescope image}
\subsection{Flat row flat pixel png format}
\subsection{Connecting theta and phi to pixels on the sky image}
\subsection{Making the blackhole red}
\subsection{Serial results}


\section{Profiling}
\subsection{Serial code profile}
\subsection{What to parallelize?}

\end{document}
